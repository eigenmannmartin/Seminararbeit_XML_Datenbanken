\documentclass[oneside,11pt,parskip=half,ngerman]{scrreprt}
%\documentclass[a4paper,oneside,11pt,parskip=half,ngerman]{scrreprt}
\usepackage{typearea}

\usepackage[ngerman]{babel}
\usepackage[colorlinks=false, pdfborder={0 0 0}]{hyperref}
\usepackage{multirow}
\usepackage{booktabs}
\usepackage[utf8]{inputenc}
\usepackage[T1]{fontenc}
\usepackage{charter}
\usepackage[expert]{mathdesign}

\usepackage{listings}
\usepackage{xcolor}
\usepackage{color}
\usepackage{fancyhdr}
\usepackage{rotating}
\usepackage{titlesec}
\usepackage{mathptmx}
\usepackage{amssymb} % checkmark
% \usepackage{helvet}
\usepackage[scaled]{uarial}
\renewcommand*\familydefault{\sfdefault} %% Only if the base font of the document is to be sans serif
\usepackage[squaren]{SIunits}
\usepackage{graphicx}
\usepackage{url}
\usepackage{geometry}
\usepackage[absolute]{textpos}
\usepackage{makeidx}
\usepackage{colortbl}
\usepackage{pdflscape}
\usepackage{pdfpages}
\usepackage{tabularx}
\usepackage{lmodern}
\usepackage{longtable}
\usepackage{array}
\usepackage{float}
\usepackage{scrhack}
\usepackage{fancyhdr}
\usepackage{fancyvrb}

\usepackage[section]{placeins}

\usepackage{pdfpages}

\usepackage[firstpage]{draftwatermark}

%Typesetting
\usepackage[activate=true,final,tracking=true,kerning=true,spacing=true,factor=1100,stretch=10,shrink=10]{microtype}
\SetProtrusion{encoding={*},family={bch},series={*},size={6,7}}
              {1={ ,750},2={ ,500},3={ ,500},4={ ,500},5={ ,500},
               6={ ,500},7={ ,600},8={ ,500},9={ ,500},0={ ,500}}

% Bibliography
\usepackage[backend=bibtex]{biblatex}
\usepackage{csquotes}
\bibliography{./Citer}


%define page margin
\geometry{a4paper, top=30mm, left=30mm, right=30mm, bottom=30mm,headsep=10mm,footskip=10mm}

\usepackage{tocloft}
\setlength\cftbeforetoctitleskip{-2.5em}
\setlength\cftbeforeloftitleskip{-2.5em}
\setlength\cftbeforelottitleskip{-2.5em}




\usepackage{graphicx}
% We will generate all images so they have a width \maxwidth. This means
% that they will get their normal width if they fit onto the page, but
% are scaled down if they would overflow the margins.
\makeatletter
\def\maxwidth{\ifdim\Gin@nat@width>\linewidth\linewidth
\else\Gin@nat@width\fi}
\makeatother
\let\Oldincludegraphics\includegraphics
\renewcommand{\includegraphics}[1]{\Oldincludegraphics[width=\maxwidth,height=20em,keepaspectratio]{#1}}


% Chapter styling
\usepackage[grey]{quotchap}
\makeatletter 
\renewcommand*{\chapnumfont}{%
  \usefont{T1}{\@defaultcnfont}{b}{n}\fontsize{80}{100}\selectfont% Default: 100/130
  \color{chaptergrey}%
}
\makeatother

\begin{titlepage}
\title{\bigskip \bigskip Herstellen einer WLAN Verbindung über grosse Distanzen}
\author{Martin Eigenmann}

%\include{../content/Titelblatt.tex}
%\includepdf[pages=1]{../content/Titelblatt}

\end{titlepage}

\begin{document}  
\maketitle

\pagenumbering{roman}

\chapter{Abstract}\label{abstract}

asdf

\setcounter{tocdepth}{1}

\chapter{Inhaltsverzeichnis}\label{inhaltsverzeichnis}

\renewcommand{\contentsname}{} 

\begingroup\let\clearpage\relax
\tableofcontents
\endgroup

\microtypesetup{protrusion=true}

\newpage

\pagenumbering{arabic}

\chapter{Einleitung}\label{einleitung}

\section{Motivation und
Fragestellung}\label{motivation-und-fragestellung}

Wer kennt das nicht, das mobile Datennetz ist langsam oder das
Datenvolumen ist bereits aufgebraucht. Im entscheidenden Moment lädt das
Bild oder Video nicht. Wäre es nicht viel angenehmer überall WLan
Empfang zu haben.

\begin{quote}
Zuhause ist, wo sich das WLan selbst verbindet.
\end{quote}

Diese Arbeit soll aufzeigen wo die Grenzen von WLan liegen und ob die
Grenzen vielleicht doch etwas entfernter liegen als intuitiv vermutet.

\section{Aufgabenstellung}\label{aufgabenstellung}

Im Folgenden ist die Aufgabenstellung gemäss EBS aufgeführt.

\subsection{Thema}\label{thema}

Ziel der Arbeit ist es ein mit WLAN weit über die spezifizierte Distanz
eine Datenverbindung auf zu bauen.

\subsection{Ausgangslage}\label{ausgangslage}

Mobile Geräte besitzen meist eine ständige Verbindung ins WWW. Ist es
aber auch möglich grosse Distanzen nur mit bekannten WLAN-Technologie zu
überwinden und so von Drittanbietern unabhängig zu werden? Welche
Voraussetzungen müssen dafür erfüllt werden?

\subsection{Ziele der Arbeit}\label{ziele-der-arbeit}

Das Ziel der Seminararbeit besteht in der Analyse der Möglichkeiten mit
bekannten WLAN-Komponenten grössere Distanzen (im Bereich von
Kilometern) zu überbrücken. Es soll theoretisch und praktisch ergründet
werden, wo die Grenzen der WLAN-Technologie liegen und ob mit leichten
Modifikationen der Sende- und Empfangs-Geräte höhere Distanzen
überbrückt werden können.

\subsection{Aufgabenstellung}\label{aufgabenstellung-1}

A1 Recherche:\\- Definition der Fachbegriffe\\- Erarbeitung der
technischen Grundlagen\\A2 Analyse:\\- Analyse der Limitationen der der
WLAN-Standards\\A3 Konzept:\\- Konzeption von Vorschlägen um die
Reichweite zu erhöhen\\A4 Umsetzung:\\- Umsetzung von zwei
Lösungsvorschlägen\\A5 Review:\\- Bewertung der umgesetzten
Lösungsvorschlägen

\subsection{Erwartete Resultate}\label{erwartete-resultate}

R1 Recherche:\\- Glossar mit Fachbegriffen\\- Erläuterung der
WLAN-Standards\\R2 Analyse:\\- Dokumentation der Limitationen im Bezug
auf die Reichweite und Übertragungsrate der WLAN-Standards.\\R3
Konzept:\\- Dokumentation der Konzepte der Lösungsvorschläge\\R4
Umsetzung:\\- Dokumentation der Umsetzung der beiden
Lösungsvorschläge\\R5 Review:\\- Dokumentation der Bewertung der
Lösungsvorschläge

\chapter{Projektplanung}\label{projektplanung}

\section{Projektplan}\label{projektplan}

Die Seminararbeit muss im Zeitraum von 04.03.2015 bis 10.06.2015
durchgeführt werden. Der praktische Teil der Arbeit wurde in der Woche
von 06.04.2015 bis 12.04.2015 durchgeführt.

\section{Zeitlicher Rahmen}\label{zeitlicher-rahmen}

Der offizielle Projektstart erfolgt mit dem Kick-Off am 04.03.2015.
Spätestens am 27.05.2015 muss ein Draft eingereicht werden. Am 17. und
18.06.2015 finden die Präsentationen statt.

Der Aufwand für die Bearbeitung der Seminararbeit soll mindestens 50
Stunden umfassen.

\section{Organisatorischer Rahmen}\label{organisatorischer-rahmen}

In der nachstehenden Tabelle sind alle massgeblich involvierten Personen
aufgeführt.

\begin{longtable}[c]{@{}ll@{}}
\caption{Involvierte Personen}\tabularnewline
\toprule
\begin{minipage}[b]{0.39\columnwidth}\raggedright\strut
Personen
\strut\end{minipage} &
\begin{minipage}[b]{0.55\columnwidth}\raggedright\strut
Kontakt
\strut\end{minipage}\tabularnewline
\midrule
\endfirsthead
\toprule
\begin{minipage}[b]{0.39\columnwidth}\raggedright\strut
Personen
\strut\end{minipage} &
\begin{minipage}[b]{0.55\columnwidth}\raggedright\strut
Kontakt
\strut\end{minipage}\tabularnewline
\midrule
\endhead
\begin{minipage}[t]{0.39\columnwidth}\raggedright\strut
\textbf{Dr.~Reto Knaack}\\(Studeingangs Leiter)
\strut\end{minipage} &
\begin{minipage}[t]{0.55\columnwidth}\raggedright\strut
ZHAW Standort Zürich\\Lagerstrasse 41 / 8004 Zürich\\Reto.Knaack@zhaw.ch
\strut\end{minipage}\tabularnewline
\begin{minipage}[t]{0.39\columnwidth}\raggedright\strut
\textbf{Peter Egli}\\(Lehrperson)
\strut\end{minipage} &
\begin{minipage}[t]{0.55\columnwidth}\raggedright\strut
ZHAW Standort Zürich\\Lagerstrasse 41 / 8004 Zürich\\eglp@zhaw.ch
\strut\end{minipage}\tabularnewline
\begin{minipage}[t]{0.39\columnwidth}\raggedright\strut
\textbf{Martin Eigenmann}\\(Student)
\strut\end{minipage} &
\begin{minipage}[t]{0.55\columnwidth}\raggedright\strut
Harfenbergstrasse 5 / 9000 St.Gallen\\study@eigenmannmartin.ch
\strut\end{minipage}\tabularnewline
\bottomrule
\end{longtable}

\chapter{Analyse}\label{analyse}

\section{Fachliche Grundlagen}\label{fachliche-grundlagen}

Mit Wireless Local Area Network, kurz WLAN, wird gemeinhin der
IEEE-802.11 assoziiert. Die Tabelle \ref{standards} zeigt die bisher
freigegebenen Standards.

\begin{longtable}[c]{@{}lll@{}}
\caption{Standards \label{standards}\\}\tabularnewline
\toprule
\begin{minipage}[b]{0.15\columnwidth}\raggedright\strut
Standard
\strut\end{minipage} &
\begin{minipage}[b]{0.20\columnwidth}\raggedright\strut
Frequenzband
\strut\end{minipage} &
\begin{minipage}[b]{0.24\columnwidth}\raggedright\strut
Datenrate (max)
\strut\end{minipage}\tabularnewline
\midrule
\endfirsthead
\toprule
\begin{minipage}[b]{0.15\columnwidth}\raggedright\strut
Standard
\strut\end{minipage} &
\begin{minipage}[b]{0.20\columnwidth}\raggedright\strut
Frequenzband
\strut\end{minipage} &
\begin{minipage}[b]{0.24\columnwidth}\raggedright\strut
Datenrate (max)
\strut\end{minipage}\tabularnewline
\midrule
\endhead
\begin{minipage}[t]{0.15\columnwidth}\raggedright\strut
802.11\\802.11b\\802.11g\\802.11a\\802.11h\\802.11n\\802.11n\\802.11ac\\802.11ad
\strut\end{minipage} &
\begin{minipage}[t]{0.20\columnwidth}\raggedright\strut
2.4 GHz\\2.4 GHz\\2.4 GHz\\5 GHz\\5 GHz\\2.4 GHz\\5 GHz\\5 GHz\\60 GHz
\strut\end{minipage} &
\begin{minipage}[t]{0.24\columnwidth}\raggedright\strut
2 Mbit/s\\6 Mbit/s\\22 Mbit/s\\22 Mbit/s\\54 Mbit/s\\450 Mbit/s\\450
Mbit/s\\660 Mbit/s\\6,7 Gbit/s
\strut\end{minipage}\tabularnewline
\bottomrule
\end{longtable}

\subsection{Antenne}\label{antenne}

Idealerweise ist eine Antenne ein Rundstrahler, welcher eine
gleichförmige Sendeleistung aufweist. Üblicherweise werden aber Antennen
verwendet, welche das Signal richten, also zum Beispiel in der
Vertikalen weniger Leistung aufweisen, in der Horizontalen dafür
weitreichender sind. So haben handelsübliche Stabantennen von Routern
eine Verstärkungswirkung um den Faktor zwei in der horizontalen Ebene.

\subsection{Sendeanlage}\label{sendeanlage}

Eine Sendeanlage umfasst die Sendeeinheit (WLAN-Karte), Antennenkabel
und Antenne. Dazugehörend sind auch anfällige Steckverbinder.

\section{Technische Limitationen}\label{technische-limitationen}

Das elektromagnetische Signal einer WLAN Anlage wird durch das
Übertragungsmedium (Luft) gedämpft. Dies wird als Freiraumdämpfung wie
folgt beschreiben wenn \(f\) die Frequenz und \(c\) die
Lichtgeschwindigkeit ist.

\(F = (\frac{4\pi r * f }{c})^2\)

Für die Frequenz 2.4GHz ergibt sich eine Freiraumdämpfungs-Kurve wie in
Abbildung \ref{fig:fa} gezeigt.

\begin{figure}[htbp]
\centering
\includegraphics{img/Freiraumdaempfung.png}
\caption{Freiraumdaempfung\label{fig:fa}}
\end{figure}

Neben der Freiraumdämpfung, vermindern auch das Signalkabel vom Sender
zur Antenne und Steckverbinder die Ausgangsleistung der gesamten Anlage.
Gängige Antennenkabel weisen eine Dämpfung von 117.9dB/100m bis
16.0dB/100m auf. Steckverbinder dämpfen zusätzlich mit 0.2dB bis 0.5dB.

Moderne Wlan-Geräte besitzen eine Empfangsempfindlichkeit von bis zu
-96dBm bei 1Mbps. Bei höheren Übertragungsraten nimmt die
Empfangsempfindlichkeit systembedingt ab. So sind bei 54Mbps bei guten
Endgeräten -73dBm zu erwarten.

Bei einer Sendeleistung von genau 20dBm (entspricht 100mW) und unter
Verwendung eines isotropen Kugelstrahler (einer idealen
Rundstrahlentanne) können bei einer Empfangsempfindlichkeit von -73dBm
maximal 443m Distanz überwunden werden. In der Praxis werden weder 20dBm
Ausgangsleistung erreicht noch existieren störungsfreie Räume.

\section{Rechtliche Limitationen}\label{rechtliche-limitationen}

Die rechtliche Beschränkungen sind je nach Einsatzgebiet unterschiedlich
und werden in der Schweiz vom Bundesamt für Kommunikation
vorgeschrieben.

\subsection{2.4 GHz Frequenzband}\label{ghz-frequenzband}

Die Leistung der gesamten Anlage ist im 2.4 GHz Band auf maximal 100mW
begrenzt. \autocite{bakomwlan}

\subsection{5 GHz Frequenzband}\label{ghz-frequenzband-1}

Die Leistung der Anlage ist für das untere 5 GHz Frequenzband (5.15 -
5.35 GHz) ist auf maximal 100mW (200mW falls die Anlage TPC unterstützt)
und für das obere 5 GHz Frequenzband (5.47 - 5.725 GHz) auf maximal
500mW (1000mW falls die Anlage TPC unterstützt) begrenzt.
\autocite{bakomwlan}

\chapter{Konzept}\label{konzept}

Im Rahmen dieser Arbeit sollen Lösungsvorschläge erarbeitet werden, die
die Reichweite von idealen Rundstrahlen weiter erhöhen können.

\section{Verminderung der Dämpfung}\label{verminderung-der-duxe4mpfung}

Um die Ausgangsleistung zu maximieren, muss die Dämpfung zwischen Sender
und Antenne minimiert werden. Dies wird duch wenige jedoch qualitativ
hochwertige Steckverbinder und sehr kurze Signalkabel sichergestellt.

\section{Erhöhung der
Sendeleistung}\label{erhuxf6hung-der-sendeleistung}

Die maximal zulässige Sendeleistung vom 20dBm wird nicht von allen
WLan-Karten untersützt. Der Ubiquiti SuperRange Cardbus kann bei einer
Übertragungsrate von 54Mbps, immer noch mit einer Sendeleistung von
20dBm operieren. Unterhalb Übertragungsraten von 24Mbps sind sogar 24dBm
möglich.\\

Zu beachten ist, dass die Ausgangsleistung (dBm) logarithmisch von der
Sendeleistung (mW) der WLan-Karte abhängig ist. So bringt doppelte
Sendeleistung nicht die doppelte Ausgangsleistung.

\hyperdef{}{richtstrahlantenne}{\section{Richtstrahlantenne}\label{richtstrahlantenne}}

Statt einen Rundstrahler zu verwenden, kann auch eine gerichtete
Verstärkung des Signals vorgenommen werden. Dazu werden typischerweise
Richstrahlantennen eingesetzt.\\Neben der Yagi-Uda-Antenne,
Wendelantenne und der Quadantenne gibt es auch
Parabolantennen.\\Yagi-Uda-Antennen erreichen eine Richtverstärkung von
3dBi bis 18dBi. Ähnlich stark sind auch Quadantennen sowie
Wendelantennen. Parabolantennen erreichen hingegen Antennengewinne von
20dBi bis weit über 50dBi hinaus. Die Signalausbreitung der
verschiedenen Antennen unterscheiden sich sehr im Öffnungswinkel
(Strahlbreite) der Hauptkeule, sowie der Aubildung von Neben- und
Rückkeulen.

Die im Rahmen dieser Arbeit verwendeten Richstrahlantennen sind eine
Yagi-Uda-Antenne mit einem Antennengewinn von 9.8 dBi bzw. 18dBi.\\

Die verwendete Parabolantenne hat einen Antennengewinn von ca. 30dBi.
(Da es sich um eine selbstgebaute Parabolantenne handelt, konnte der
Antennengewinn nicht genau ermittelt werden, da die dafür benötigten
Messinstrumente nicht zu Verfügung standen.)\\

\chapter{Umsetzung}\label{umsetzung}

\section{Vorgehen und Messungen}\label{vorgehen-und-messungen}

Es wird nur eine Messung durchgeführt falls auch eine Verbindung
hergestellt werden kann. Zur Messung der Signalstärke wird das
Linux-Tool \(wavemon\) verwendet.

\section{Richtstrahlantenne}\label{richtstrahlantenne-1}

Um das WLAN-Signal zu verstärken, wird eine Yagi-Uda-Antenne mit einem
Reflektor, einem Signalgeber und fünf Direktoren verwendet. (Siehe
Abbildung \ref{fig:routeryagi}) Dabei handelt es sich um einen Eigenbau.
Bauanleitungen mit detaillierten Beschreibungen und
Hintergrundinformationen sind online verfügbar. \autocite{eigenbauyagi}

\begin{figure}[htbp]
\centering
\includegraphics{img/router-yagi.jpg}
\caption{RouterYagi\label{fig:routeryagi}}
\end{figure}

Der Antennengewinn liegt bei 9.8dBi. Bei einer Entfernung von 200 Metern
und einer Ausgangsleistung von 16dB, ist mit -60dBm Empfangslevel zu
rechnen.

Bei einer Distanz von 200 Metern ist ohne ein modifiziertes Gegenstück
also immer noch eine sehr gute Verbindung erreichbar. Mit einem
handelsüblichen Notebook sind genau -60dBm gemessen worden. (dazu
Abbildung \ref{fig:router-yagi-reg})

\section{Richtstrahlantenne II}\label{richtstrahlantenne-ii}

Die nächst grössere Distanz, die überwunden werden soll, beträgt mehr
als 1100 Meter. Da mit der selbst gebauten Yagi-Uda-Antenne kein
Verbindungsaufbau möglich war, ist eine bessere Antenne nötig.

Mit zwei aufeinander ausgerichteten Yagi-Uda-Antennen vom Typ ABAKS
YAGI-18 die mit jeweils 15 Direktoren einen Antennengewinn von 18dBi
aufweisen und einer Sendeleistung von 21.1dB bzw. 22.3dB ist
idealerweise mit einem Empfangslevel von -60dBm zu rechnen.

Die Sendeleistung von 21.1dB bzw. 22.3dB ergibt sich aus den 24dB
maximaler Sendeleistung der WLAN-Karte abzüglich der 2.9dB bzw. 1.7dB
Dämpfung für Kabel und Verbinder.

\begin{figure}[htbp]
\centering
\includegraphics{img/yagi-wald.jpg}
\caption{ABAKS YAGI-18 im Wald\label{fig:yagi-18}}
\end{figure}

Das gemessene Empfangslevel von -80dBm (dazu Abbildung
\ref{fig:yagi-reg}) bzw. die Differenz von 20dBm zu dem erwarteten
Ergebniss, ist den störenden Objekten im Funkpfad geschuldet.

\section{Parabolspiegel}\label{parabolspiegel}

Die grösste mögliche Distanz mit Sichtverbindung, welche im Rahmen
dieser Arbeit betrachtet wird, beträgt bei 8340 Metern.

\begin{figure}[htbp]
\centering
\includegraphics{img/parabol-point.jpg}
\caption{Parabolantenne ausgerichtet auf das 8340 Meter entfernte
Gegenstück\label{fig:parabol-point}}
\end{figure}

Die WLAN-Parabolantenne ist eine Improvisation um die Richtwirkung der
Yagi-Uda-Antenne weiter zu erhöhen. Es handelst sich dabei um einen
Handelsüblichen Parabolspiegel, der auch für Satellitenfernsehen
eingesetzt wird. Der Antennengewinn beträgt ca. 30dBi. (Abbildung
\ref{fig:parabolantenne})

\begin{figure}[htbp]
\centering
\includegraphics{img/parabol.png}
\caption{Parabolantenne\label{fig:parabolantenne}}
\end{figure}

Mit der Parabolantenne mit einer Sendeleistung 22.3dB und der darauf
ausgerichteten Yagi-Uda-Antenne mit einem Antennengewinn von 18dBi und
einer Sendeleistung von 21.1dB ist idealerweise ein Empfangslevel von
-66dBm erreichbar.

Der gemessene Signalpegel beträgt -67dBm. (dazu Abbildung
\ref{fig:parabolantenne-reg})

\chapter{Review und Bewertung}\label{review-und-bewertung}

Die 3 umgesetzten Antennenkonzepte erfüllen unterschiedliche
Anforderungen. Bewertet werden die Konzepte und deren Umsetzung deshalb
in Bezug auf Reichweite, Handhabbarkeit, Einsatzgebiet und Kosten.

\section{Richtstrahlantenne}\label{richtstrahlantenne-2}

Die Reichweite verdoppelt sich auf \textasciitilde{} 200 Meter. Der Bau
sowie die Installation erfordern kein spezifisches Fachwissen und sind
sehr einfach und schnell zu erledigen; Sofern der Router über externe
Antennen verfügt.\\Typischerweise können so Funklöcher innerhalb einer
Wohnung oder Wohnkomplex abgedeckt werden.\\Die Kosten belaufen sich auf
wenige Rappen bis Franken.

\section{Richtstrahlantenne II}\label{richtstrahlantenne-ii-1}

Die Reichweite steigert sich enorm. Im Test war über 1 Kilometer
Funkstrecke überbrückbar. Die Installation erfordert kein spezifisches
Fachwissen, lediglich die Ausrichtung der Antennen muss sehr sorgfältig
durchgeführt werden. Diese Art von Richtstrahlantennen setzen ausserdem
voraus dass auf Empfänger- und Sender-Seite Modifikationen vorgenommen
werden und fordern daher einen grösseren Aufwand bei der
Installation.\\Typischerweise wird mit Leistungsstarken
Yagi-Uda-Antennen ein Kommunikationskanal über Gebäudegrenzen hinweg
realisiert.\\Die Kosten belaufen sich auf \textasciitilde{}100 CHF.

\section{Parabolspiegel}\label{parabolspiegel-1}

Die Reichweite scheint fast unbegrenzt. Im Test konnten mehr als 8
Kilometer Funkstrecke überwunden werden. Die Installation erfordert
Fachwissen und Geduld bei der Ausrichtung der Antennen. Üblicherweise
werden auch Sendekarten mit höheren Ausgangsleistungen verwendet, um so
an der Antenne die maximal erlaubten 20dB zu erreichen.\\Parabolantennen
werden nur selten im zivilen Umfeld verwendet und so müssen nur
vereinzelt unzugängliche Ortschaften so erschlossen werden.\\Die Kosten
bewegen sich je nach verwendetem Parabolspiegel und Antenne im gehobenen
drei stelligen Franken Bereich.

\section{Bewertung}\label{bewertung}

Sowohl bezüglich der Leistung als auch der Handhabung überzeugt die
eingesetzte Yagi-Uda-Antenne. Einfache Installation und grossartige
Leistung um grosse Distanzen zu überbrücken runden das Paket
hab.\\Selbstgebaute Yagi-Uda-Antennen überzeugen zwar für den
Heimgebrauch. Ihre Leistungsfähigkeit ist für einen professionellen
Gebrauch ist jedoch ungeeignet.\\Grosse Parabolantennen überzeugen zwar
in Puncto Leistung, haben jedoch bei Handhabung und Preis deutliche
Defizite.

\chapter{Fazit und Schlusswort}\label{fazit-und-schlusswort}

Bereits mit einfachen und kosteneffizienten Mitteln (siehe. Kapitel
\hyperref[richtstrahlantenne]{Richtstrahlantenne}) kann die Reichweite
von handelsüblichen WLAN-Routern erhöht werden. Um die Reichweite in der
nahen Umgebung gezielt zu erhöhen, und so z.B. den Empfang im Garten zu
verbessern, sind selbst gebaute Yagi-Uda-Antenne eine Überlegung wert.

Der Einsatz von grossen Yagi-Uda-Antennen oder sogar Parabolantennen
erhöht die Reichweite enorm. So konnte eine Verbindung über die Distanz
von 8340 Metern hergestellt werden. Unter Verwendung zweier
Parabolantennen wäre eine noch deutlich grössere Distanz vorstellbar.
Sowohl der Einsatz von kommerziellen Yagi-Uda-Antennen als auch
Parabolantennen ist sehr aufwendig und bedarf einiger Einarbeitung in
das Fachgebiet. Auch unterscheidet sich hier das Einsatzgebiet deutlich
vom \enquote{Home and Entertainment} Sektor und bedient somit seltene
Sonder-Anforderungen.

Der Student konnte zeigen, dass mit handelsüblichen WLAN-Geräten und
etwas Draht die Reichweite von WLAN-Signalen deutlich erhöht werden
konnte. Unter dem Einsatz von kommerziellen Richtstrahlentannen können
Distanzen von über einem Kilometer überwunden werden.\\Mit etwas
handwerklichem Geschick kann mit einer Satellitenschüssel und einer
Richtstrahlentanne eine Parabolantenne gebaut werden, die
Signalreichweiten von über 8 Kilometern zulassen.

\appendix

\chapter{Appendix}\label{appendixA}

\section{Wavemon}\label{wavemon}

\begin{figure}[htbp]
\centering
\includegraphics{img/router-yagi-reg.jpg}
\caption{Screenshot: Wavemon selbstgebaute
YAGI\label{fig:router-yagi-reg}}
\end{figure}

\begin{figure}[htbp]
\centering
\includegraphics{img/yagi-reg.jpg}
\caption{Screenshot: Wavemon ABAKS YAGI-18\label{fig:yagi-reg}}
\end{figure}

\begin{figure}[htbp]
\centering
\includegraphics{img/parabol-reg.jpg}
\caption{Screenshot: Wavemon
Parabolantenne\label{fig:parabolantenne-reg}}
\end{figure}

\section{Glossar}\label{glossar}

\textbf{TPC (Transmit Power Control)}\\TPC ist eine Funktionalität, die
es der Sendeeinheit erlaubt, die Sendeleistung dynamisch anzupassen.

\textbf{Freiraumdämpfung}\\Die Freiraumdämpfung beschreibt die
Abschwächung von elektromagnetischen Signalen im freien Raum.

\textbf{Antennengewinn}\\Der Antennengewinn beschreibt die Verstärkung
des Signals in eine bestimmte Richtung und wird in dBi angegeben.

\section{Quellenverzeichnis}\label{quellenverzeichnis}

\vspace*{-2.5cm}\renewcommand{\bibname}{}\begingroup \let\clearpage\relax
\printbibliography
\endgroup

\section{Tabellenverzeichnis}\label{tabellenverzeichnis}

\renewcommand{\listtablename}{} 

\begingroup \let\clearpage\relax
\listoftables
\endgroup

\section{Abbildungsverzeichnis}\label{abbildungsverzeichnis}

\renewcommand{\listfigurename}{} 

\begingroup\let\clearpage\relax
\listoffigures
\endgroup


\end{document}